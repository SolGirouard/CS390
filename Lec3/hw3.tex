\documentclass[12pt]{article}
\usepackage{amsfonts,amsmath,amssymb,graphicx,url}

% Old Stuff
%%\oddsidemargin=0.15in
%%\evensidemargin=0.15in
%%\topmargin=-.5in
%%\textheight=9in
%%\textwidth=6.25in

\setlength{\oddsidemargin}{.25in}
\setlength{\evensidemargin}{.25in}
\setlength{\textwidth}{6.25in}
\setlength{\topmargin}{-0.4in}
\setlength{\textheight}{8.5in}

\DeclareMathOperator*{\Expect}{\mathbb{E}}

\newcommand{\heading}[5]{
   \renewcommand{\thepage}{#1-\arabic{page}}
   \noindent
   \begin{center}
   \framebox{
      \vbox{
    \hbox to 6.2in { {\bf CS390 Computational Game Theory and Mechanism Design}
     	 \hfill #2 }
       \vspace{4mm}
       \hbox to 6.2in { {\Large \hfill #5  \hfill} }
       \vspace{2mm}
       \hbox to 6.2in { {\it #3 \hfill #4} }
      }
   }
   \end{center}
   \vspace*{4mm}
}

\newcommand{\handout}[3]{\heading{#1}{#2}{Yiqing Hua}{}{#3}}

\setlength{\parindent}{0in}
\setlength{\parskip}{0.1in}

\begin{document}
\handout{1}{July 8, 2013}{Problem Set 3}

\paragraph{Problem 1} (No collaborator.)

The Nash equilibria are still two, the one that all choose the lower route,
and the one that only one choose the upper route. 
Obviously, the first one has worse objective funcion which is $1$ no matter what value $d$ holds. \\
To compute the price of anarchy, we need to calculate the cost under the best stuation.
Suppose there are $k$ person choose the lower route, 
let $c$ be the value of the objective function.
\begin{align*}
    c &= \frac{n - k + (\frac{k}{n})^dk}{n}  \\
      &= \frac{n - k + k^{d + 1}/n^d}{n} \\
    c' &= -\frac{1}{n} + (d + 1)k^d/n^{d + 1} 
\end{align*}
To make $c$ be the minimum, the derivetive should be $0$. 
\begin{align*}
	k &= \frac{n}{(d + 1)^{1/d}} \\ 
	c &= 1 - (d + 1)^{-1/d} + (d + 1)^{-1-1/d}  \\
	  &= 1 - (1 + d)^{1/d \cdot -1} + (1 + d)^{1/d \cdot -(d + 1)}
\end{align*}
When $n$ goes to infinity, PoA goes to $\frac{1}{1 - (d + 1)^{-1/d} + (d + 1)^{-1-1/d}}$. \\
When $d$ goes to infinity,  
$c$ goes to $0$.\\
Then PoA goes to infinity.

\bigskip

\paragraph{Problem 2} (No collaborator.)

Suppose the strategy set having the best sum of utility is $S$,
and the strategy set of the NE having the worst objective function  is $S'$.\\
Consider a single player $i$, 
if the absolute value of his utility in the NE exceed n times his utility in the $S$,
then he can choose his route in $S$, 
since no matter how many people choose the same route as him,
his cost on this route won't exceed $n$ times any possible cost on this route. \\
Then for any player, in NE, 
the absolute value of his cost won't exceed $n$ times that of his cost in $S$.
Then the absolute value of total cost in NE won't exceed $n$ times that of the best objective function.\\
Therefore, the price of anarchy won't exceed $1$.


\bigskip

\paragraph{Problem 3} (No collaborator.)

The minimum value of the objective function is $n$.
Since no matter how the player chooses his strategy, 
mixed or pure, his utility is always larger than or equal to $1$.
And when we let player $i$ choose the ith machine, we can reach the objective function $n$.\\
The pure NE is the permutations of $n$, and they all have the objective funcion of $n$.
So we'll only consider the only mixed Nash Equilibrium,
that $\forall i$, $\sigma_i = \frac{1}{n}M_1 + \frac{1}{n}M_2 \ldots + \frac{1}{n}M_n$.\\
We'll use the indicator function,
\begin{align*}
X_{ij} = 
\begin{cases}
            1 &\mbox{$s_i = M_j$, $\forall M_j$} \\
            0 &\mbox{otherwise} \\
         \end{cases}
\end{align*}
\begin{align*}
\Expect_{s_i \sim \sigma_i} c_{ij} 
    &= 1 + \Sigma_{k \neq i, s_k \sim \sigma_i}\Expect X_{kj} \\
    &= 1 + \Sigma_{k \neq i}\frac{1}{n} \cdot 1 + (1 - \frac{1}{n}) \cdot 0\\
    &= 1 + \frac{n - 1}{n} \\
    &= 2 - \frac{1}{n} \\
\Expect c_i &= \Sigma_{k = 1}^n \frac{1}{n}c_{ik} \\
            &= 2 - \frac{1}{n}\\
f(\sigma) &= \Sigma_{i = 1}^n c_i \\
          &= n \cdot (2 - \frac{1}{n}) \\
PoA &=  2 - \frac{1}{n}
\end{align*}



\bibliographystyle{agsm}

\begin{thebibliography}{99}

\bibitem{OR94}{M. J. Osborne and A. Rubinstein. {\em A course in game theory.} MIT Press, 1994.}

\bibitem{NRTV07}{N. Nisan, T. Roughgarden, E. Tardos, and V. Vazirani (eds). {\em Algorithmic game theory.} Cambridge University Press, 2007. (Available at \url{http://www.cambridge.org/journals/nisan/downloads/Nisan_Non-printable.pdf}.)}

\end{thebibliography}

\end{document}








