\documentclass[12pt]{article}
\usepackage{amsfonts,amsmath,amssymb,graphicx,url}

% Old Stuff
%%\oddsidemargin=0.15in
%%\evensidemargin=0.15in
%%\topmargin=-.5in
%%\textheight=9in
%%\textwidth=6.25in

\setlength{\oddsidemargin}{.25in}
\setlength{\evensidemargin}{.25in}
\setlength{\textwidth}{6.25in}
\setlength{\topmargin}{-0.4in}
\setlength{\textheight}{8.5in}

\newcommand{\heading}[5]{
   \renewcommand{\thepage}{#1-\arabic{page}}
   \noindent
   \begin{center}
   \framebox{
      \vbox{
    \hbox to 6.2in { {\bf CS390 Computational Game Theory and Mechanism Design}
     	 \hfill #2 }
       \vspace{4mm}
       \hbox to 6.2in { {\Large \hfill #5  \hfill} }
       \vspace{2mm}
       \hbox to 6.2in { {\it #3 \hfill #4} }
      }
   }
   \end{center}
   \vspace*{4mm}
}

\newcommand{\handout}[3]{\heading{#1}{#2}{Yiqing Hua}{}{#3}}

\setlength{\parindent}{0in}
\setlength{\parskip}{0.1in}

\begin{document}
\handout{1}{July 8, 2013}{Problem Set 3}

\paragraph{Problem 1} %(Collaborated with AAA, BBB, and CCC.)

The Nash equilibria are still two, the one that all choose the lower route,
and the one that only one choose the upper route. 
Obviously, the first one has worse objective funcion which is $1$ no matter what value $d$ holds. \\
To compute the price of anarchy, we need to calculate the cost under the best stuation.
Suppose there are $k$ person choose the lower route, 
let $c$ be the value of the objective function.
\begin{align*}
    c &= \frac{n - k + (\frac{k}{n})^dk}{n}  \\
      &= \frac{n - k + k^{d + 1}/n^d}{n}
    \frac{dc}{dk} &= -\frac{1}{n} + (d + 1)k^d/n^{d + 1} 
\end{align*}
To make $c$ be the minimum, the derivitive should be $0$. 
\begin{align*}
    k &= n - (\frac{n}{2})^{\frac{d}{2d-1}} \\
    c &= 
\end{align*}
When $n$ goes to infinity, $d > 1$, $\frac{1 - d}{2d - 1} < 0$ then $c$ goes to $1$.
Then PoA is 1. \\
When $d$ goes to infinity, 

\bigskip

\paragraph{Problem 2} (Collaborated with YYY and ZZZ.)

Your solution goes here.

\bigskip

\paragraph{Problem 3} (No collaborator.)

Your solution goes here.

\bibliographystyle{agsm}

\begin{thebibliography}{99}

\bibitem{OR94}{M. J. Osborne and A. Rubinstein. {\em A course in game theory.} MIT Press, 1994.}

\bibitem{NRTV07}{N. Nisan, T. Roughgarden, E. Tardos, and V. Vazirani (eds). {\em Algorithmic game theory.} Cambridge University Press, 2007. (Available at \url{http://www.cambridge.org/journals/nisan/downloads/Nisan_Non-printable.pdf}.)}

\end{thebibliography}

\end{document}








