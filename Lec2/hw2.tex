\documentclass[12pt]{article}
\usepackage{amsfonts,amsmath,amssymb,graphicx,url}

% Old Stuff
%%\oddsidemargin=0.15in
%%\evensidemargin=0.15in
%%\topmargin=-.5in
%%\textheight=9in
%%\textwidth=6.25in

\setlength{\oddsidemargin}{.25in}
\setlength{\evensidemargin}{.25in}
\setlength{\textwidth}{6.25in}
\setlength{\topmargin}{-0.4in}
\setlength{\textheight}{8.5in}

\newcommand{\heading}[5]{
   \renewcommand{\thepage}{#1-\arabic{page}}
   \noindent
   \begin{center}
   \framebox{
      \vbox{
    \hbox to 6.2in { {\bf CS390 Computational Game Theory and Mechanism Design}
     	 \hfill #2 }
       \vspace{4mm}
       \hbox to 6.2in { {\Large \hfill #5  \hfill} }
       \vspace{2mm}
       \hbox to 6.2in { {\it #3 \hfill #4} }
      }
   }
   \end{center}
   \vspace*{4mm}
}

\newcommand{\handout}[3]{\heading{#1}{#2}{Yiqing Hua}{}{#3}}

\setlength{\parindent}{0in}
\setlength{\parskip}{0.1in}

\begin{document}
\handout{1}{July 3, 2013}{Problem Set 2}

\paragraph{Problem 1} (No collaborator.)

First we'll prove that $D$ can never be a best response for any $\sigma_{-3} \in \delta(S_{-3})$. \\
If $D$ is a best response for some $\sigma_{-3} = (k_1L+(1-k_1)R, k_2U+(1-k_2)D)$.
Then it should satisfy that,
\begin{align*}
    u_3(A, \sigma_{-3}) = 9k_1k_2 &\leq 6k_1k_2 + 6(1-k_1)(1-k_2) = u_3(D, \sigma_{-3}) \\
    u_3(B, \sigma_{-3}) &= 9k_1(1 - k_2) + 9k_2(1 - k_1) \leq u_3(D, \sigma_{-3}) \\
    u_3(C, \sigma_{-3}) &= 9(1 - k_1)(1 - k_2) \leq u_3(D, \sigma_{-3}) \\
    0 \leq &k_1 \leq 1 \\
    0 \leq &k_2 \leq 1
\end{align*}
By using mathematica, we'll find that there is no $k_1, k_2$ satisfies the above inequailities.
Thus, there doesn't exist a situation such that $D$ is the best response for it. \\
If $D$ is strictly dominated, then it is a never best response. 
Therefore for any $\sigma_{-3} \in \delta(S_{-3})$, $\sigma_3 \in B_3(\sigma_{-3})$, $D \notin \sigma_3$. \\
For $\sigma_{-3} = (\frac{1}{2}L + \frac{1}{2}R , \frac{1}{2}U + \frac{1}{2}D)$.
We can easily find that $u_3(\frac{1}{2}B + \frac{1}{2}D, \sigma_{-3}) = 7.5$. 
Then we'll prove that for any mixed strategy that does not contain $D$ may acheive a less utility than $7.5$.\\
For any mixed strategy that does not contain $D$, assume it's $\sigma_3' = aA + bB + (1 - a - b)C$, 
$0 \leq a \leq 1$, $0 \leq b \leq 1$.
\begin{align*}
u_3(\sigma_3', \sigma_{-3}) &= \frac{9}{4}a + \frac{9}{2}b + \frac{9}{4}(1 - a - b) \\
                            &= \frac{9}{4} + \frac{9}{4}b \leq 4.5 < 7.5
\end{align*}
Thus any mixed strategy taht does not contain $D$ is not the best response for $\sigma_{-3}$.\\
Therefore, $D \in B_3(\sigma_{-3})$, and it is not a never best response. 
So, $D$ is not strictly dominated.




\bigskip

\paragraph{Problem 2} (No collaborator.)

We'll construct a game played by only one player.
He can choose whether to play it or terminate it.
The utility of terminating the game is 0, and that of playing it infinitely is 1. \\
For the profile that he always terminates the game,
for any nodes in the game tree, if he chooses the other strategy that he'll continue to play it, 
he'll still get the utility of 0, since he'll terminate it in the next step.
But this probile is obviously not the SPE.
So this is a game that not do not hold the one deviation property.


\bigskip

\paragraph{Problem 3} (No collaborator.)

Consider the situation that if there are only two persons left, D and E. 
Since whatever proposal D gives, he himself will accept it and whatever E agrees on that proposal, they'll have a tie. 
So D can distribute in any way to maximize his utility,
he'll give 100 to himself and 0 to E. \\
When there are three persons left C, D, E.
Then whatever proposal C gives, if he gives E more than 1 coin.
E will accept the proposal since C and E can make the majority and E will get more than if they throw C out of the deck.
Therefore, to maximize his own utility, 
E may give E 1 coin and give C nothing while he himself has the other 99.  \\
Then we'll consider the situation when there are B, C, D, E left.
If anyone other than B accepts his proposal, B can survive.
Then if B gives 1 coin to D, D can get more than when B is thrown overboard.
So for B, his best choice is to give D 1 coin and the others nothing while he can get 99 coins. \\
Then, finally, we'll consider the initial state.
At this moment, C and E will find that if A gives them some profits, 
it's better for them to let A survive.
So for A, his best choice is to give C and E 1 coin each and others nothing while he can get 98 coins left.
This is the final distribution of the coins.

\bibliographystyle{agsm}

\begin{thebibliography}{99}


\bibitem{OR94}{M. J. Osborne and A. Rubinstein. {\em A course in game theory.} MIT Press, 1994.}

\bibitem{NRTV07}{N. Nisan, T. Roughgarden, E. Tardos, and V. Vazirani (eds). {\em Algorithmic game theory.} Cambridge University Press, 2007. (Available at \url{http://www.cambridge.org/journals/nisan/downloads/Nisan_Non-printable.pdf}.)}

\end{thebibliography}

\end{document}








