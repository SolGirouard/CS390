\documentclass[12pt]{article}
\usepackage{amsfonts,amsmath,amssymb,graphicx,url}
\usepackage{fullpage}


\newtheorem{definition}{Definition}
\newtheorem{remark}{Remark}
\newtheorem{theorem}{Theorem}
\newtheorem{lemma}[theorem]{Lemma}
\newtheorem{corollary}[theorem]{Corollary}
\newtheorem{proposition}[theorem]{Proposition}
\newtheorem{claim}[theorem]{Claim}
\newtheorem{observation}{Observation}


\newenvironment{proof}{\noindent{\em Proof.} \hspace*{1mm}}{
	\hspace*{\fill} $\Box$ }
\newenvironment{proof_of}[1]{\noindent {\bf Proof of #1:}
	\hspace*{1mm}}{\hspace*{\fill} $\Box$ }
\newenvironment{proof_claim}{\begin{quotation} \noindent}{
	\hspace*{\fill} $\diamond$ \end{quotation}}

\setlength{\oddsidemargin}{.25in}
\setlength{\evensidemargin}{.25in}
\setlength{\textwidth}{6.25in}
\setlength{\topmargin}{-0.4in}
\setlength{\textheight}{8.5in}

\newcommand{\heading}[5]{
   \renewcommand{\thepage}{#1-\arabic{page}}
   \noindent
   \begin{center}
   \framebox{
      \vbox{
    \hbox to 6.2in { {\bf CS390 Computational Game Theory and Mechanism Design}
     	 \hfill #2 }
       \vspace{4mm}
       \hbox to 6.2in { {\Large \hfill #5  \hfill} }
       \vspace{2mm}
       \hbox to 6.2in { {\it #3 \hfill #4} }
      }
   }
   \end{center}
   \vspace*{4mm}
}

\newcommand{\handout}[3]{\heading{#1}{#2}{Scribed by Yiqing Hua}{}{#3}}

\setlength{\parindent}{0in}
\setlength{\parskip}{0.1in}

\begin{document}
\handout{1}{July 1, 2013}{Lecture 1, Part 2}
In this class we talked about iterated elimination.   \\
In a strategic game, 
if one pure strategy strictly dominates the others then it is obvious the strategy we seek for. \\
We'll talk about the notion of strictly dominence, 
and the process to eliminate the strategies strictly dominated by the others
so that none of the players may choose them.

\section{Definitions}

Iterated elimination is defined as follows.

\begin{definition}\label{def:normal}
    In a strategic game $\langle N, S, u \rangle$,
    a set $X \subseteq S$ survives iterated elimination of strictly dominated strategy of $X = X_1 \times X_2 \dots \times X_n$, if $\exists$ a finite sequence $S^0, S^1 \dots S^K$ s.t 
    \begin{itemize}
        \item $S^0 = S$
        \item $S^k = S^k_1 \times S^k_2 \dots S^k_n$
        \item $S^K = X$
        \item $\forall 0 \leq k \leq K$, 
              $S^{k + 1}_i \subseteq S^k_i$, 
              and $S^k \backslash S^{k + 1} \neq \emptyset$
        \item $\forall i$, $s_i \in S^k_i \backslash S^{k + 1}_i$,
              if $\exists \sigma_i \in \delta(S^k_i),$ s.t
              $u_i(\sigma_i, S_{-i}) > u_i(s_i, S_{-i})$
               $\forall S_{-i} \in S_{-i}^k.$
              For $G^k = \langle N, S^k, u \rangle$,
              we say that $s_i$ is strictly dominated by $\sigma_i$ in $G^k$, that $s_i \prec \sigma_i$.
        \item $\forall i$, 
              $S_i^K$ contains no strategy that is strictly dominated over $S^K$.
    \end{itemize}
\end{definition}

\paragraph{Remark.}
    Note that $\sigma_i$ can be either pure strategy or mixed strategy 
    while $s_i \in S_i^k$ are all pure strategies.


\section{Example}
Given the game $\langle \{1,2\}, \{T, M, B\} \times \{L, R\}, u\rangle$.
The utility is given as the following table.
\begin{table}[hbtp]
  \centering
  \begin{tabular}{|c|c|c|}
    \hline
    % after \\: \hline or \cline{col1-col2} \cline{col3-col4} ...
         & L     & R \\ \hline
    T    & (3, 0)& (0, 1) \\ \hline
    M    & (0, 0)& (3, 1) \\ \hline  
    B    & (1, 1)& (1, 0) \\ \hline
  \end{tabular}
  \caption{The utility of the given game.}\label{tab:1}
\end{table}
The iterated elimination acts as follows. \\
\begin{itemize}
    \item
    First we have $S^0 = S = \{T, M, B\} \times \{L, R\}$ \\
    Since no pure strategy can dominate any strategy of each player, 
    we should look for some mixed strategy to do our elimination. \\
    We choose $\sigma_1 = \frac{1}{2}M + \frac{1}{2}T$.
    Then we have 
    \begin{align*}
        u_1(\sigma_1, L) = 1.5 > 1 = u_1(B, L) \\
        u_1(\sigma_1, R) = 1.5 > 1 = u_1(B, R)
    \end{align*}
    Therefore, $B$ is strictly dominated by $\sigma_1$.
    We can get $S^1$.
    \item
    $S^1 = \{T, M\} \times \{L, R\}$ \\
    Consider player 2, we have 
    \begin{align*}
        u_2(T, R) = 1 > 0 = u_2(T, L)\\
        u_2(M, R) = 1 > 0 = u_2(M, L)
    \end{align*}
    Therefore, $L$ is strictly dominated by $R$.
    We now have $S^2$.
    \item
    $S^2 = \{T, M\} \times \{R\}$ \\
    Since $u_1(M, R) = 3 > 0 = u_1(T, R)$,
    we can now eliminate strategy $T$.
    \item
    And now we have the final state $S^3 = \{M\} \times \{R\}$.
\end{itemize}


%\begin{figure}
%  \centering
%  % Requires \usepackage{graphicx}
%  \includegraphics[width=0.3\textwidth]{x.pdf}\\
%  \caption{A figure}\label{fig:1}
%\end{figure}


\bibliographystyle{agsm}

\begin{thebibliography}{99}

\bibitem{OR94}{M. J. Osborne and A. Rubinstein. {\em A course in game theory.} MIT Press, 1994.}

\bibitem{NRTV07}{N. Nisan, T. Roughgarden, E. Tardos, and V. Vazirani (eds). {\em Algorithmic game theory.} Cambridge University Press, 2007. (Available at \url{http://www.cambridge.org/journals/nisan/downloads/Nisan_Non-printable.pdf}.)}
    
\end{thebibliography}

\end{document}








