\documentclass[12pt]{article}
\usepackage{amsfonts,amsmath,amssymb,graphicx,url}
\usepackage{fullpage}


\newtheorem{definition}{Definition}
\newtheorem{remark}{Remark}
\newtheorem{theorem}{Theorem}
\newtheorem{lemma}[theorem]{Lemma}
\newtheorem{corollary}[theorem]{Corollary}
\newtheorem{proposition}[theorem]{Proposition}
\newtheorem{claim}[theorem]{Claim}
\newtheorem{observation}{Observation}


\newenvironment{proof}{\noindent{\em Proof.} \hspace*{1mm}}{
	\hspace*{\fill} $\Box$ }
\newenvironment{proof_of}[1]{\noindent {\bf Proof of #1:}
	\hspace*{1mm}}{\hspace*{\fill} $\Box$ }
\newenvironment{proof_claim}{\begin{quotation} \noindent}{
	\hspace*{\fill} $\diamond$ \end{quotation}}

\setlength{\oddsidemargin}{.25in}
\setlength{\evensidemargin}{.25in}
\setlength{\textwidth}{6.25in}
\setlength{\topmargin}{-0.4in}
\setlength{\textheight}{8.5in}

\newcommand{\heading}[5]{
   \renewcommand{\thepage}{#1-\arabic{page}}
   \noindent
   \begin{center}
   \framebox{
      \vbox{
    \hbox to 6.2in { {\bf CS390 Computational Game Theory and Mechanism Design}
     	 \hfill #2 }
       \vspace{4mm}
       \hbox to 6.2in { {\Large \hfill #5  \hfill} }
       \vspace{2mm}
       \hbox to 6.2in { {\it #3 \hfill #4} }
      }
   }
   \end{center}
   \vspace*{4mm}
}

\newcommand{\handout}[3]{\heading{#1}{#2}{Scribed by Yiqing Hua}{}{#3}}

\setlength{\parindent}{0in}
\setlength{\parskip}{0.1in}

\begin{document}
\handout{1}{July 1, 2013}{Lecture 1, Part 1}
In this class we gave an introduction to the game theory, which consists of two parts, Game Analysis and Mechanism Design. We talk about normal form games and which kind of situation it will finally be in. 

\section{Definitions}

Normal-form games are defined as follows.\\

\begin{definition}\label{def:normal}
    A of a game(a strategic game) is a triple $(N, S, u)$. \\
    \begin{itemize}
    \item $N = \{1, 2, 3, ...n\}$ is the labels of n players.\\
    \item $S = S_1 \times S_2 \times ... \times S_n$ 
          $S_i$ is the pure strategy set of each player i.\\
          $s = (s_1, s_2, ..., s_n)$ is the strategy profile.
          $s_1 \in S_1, s_2 \in S_2, ... ,s_n \in S_n$
    \item $u = $
    \end{itemize}


...
\end{definition}

\begin{theorem}\label{thm:test}
Normal-form games are cool.
\end{theorem}
\begin{proof}
Thus Theorem \ref{thm:test} holds.
\end{proof}

\begin{table}[hbtp]
  \centering
  \begin{tabular}{|c|c|}
    \hline
    % after \\: \hline or \cline{col1-col2} \cline{col3-col4} ...
    Left & 0,1 \\ \hline
    Right & 1,0 \\ \hline
  \end{tabular}
  \caption{A table}\label{tab:1}
\end{table}

\paragraph{Remark.}...

%\begin{figure}
%  \centering
%  % Requires \usepackage{graphicx}
%  \includegraphics[width=0.3\textwidth]{x.pdf}\\
%  \caption{A figure}\label{fig:1}
%\end{figure}


\bibliographystyle{agsm}

\begin{thebibliography}{99}

\bibitem{OR94}{M. J. Osborne and A. Rubinstein. {\em A course in game theory.} MIT Press, 1994.}

\bibitem{NRTV07}{N. Nisan, T. Roughgarden, E. Tardos, and V. Vazirani (eds). {\em Algorithmic game theory.} Cambridge University Press, 2007. (Available at \url{http://www.cambridge.org/journals/nisan/downloads/Nisan_Non-printable.pdf}.)}
    
\end{thebibliography}

\end{document}








