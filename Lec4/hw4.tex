\documentclass[12pt]{article}
\usepackage{amsfonts,amsmath,amssymb,graphicx,url}

% Old Stuff
%%\oddsidemargin=0.15in
%%\evensidemargin=0.15in
%%\topmargin=-.5in
%%\textheight=9in
%%\textwidth=6.25in

\setlength{\oddsidemargin}{.25in}
\setlength{\evensidemargin}{.25in}
\setlength{\textwidth}{6.25in}
\setlength{\topmargin}{-0.4in}
\setlength{\textheight}{8.5in}

\newcommand{\heading}[5]{
   \renewcommand{\thepage}{#1-\arabic{page}}
   \noindent
   \begin{center}
   \framebox{
      \vbox{
    \hbox to 6.2in { {\bf CS390 Computational Game Theory and Mechanism Design}
     	 \hfill #2 }
       \vspace{4mm}
       \hbox to 6.2in { {\Large \hfill #5  \hfill} }
       \vspace{2mm}
       \hbox to 6.2in { {\it #3 \hfill #4} }
      }
   }
   \end{center}
   \vspace*{4mm}
}

\newcommand{\handout}[3]{\heading{#1}{#2}{Yiqing Hua}{}{#3}}

\setlength{\parindent}{0in}
\setlength{\parskip}{0.1in}

\begin{document}
\handout{1}{July 10, 2013}{Problem Set 4}

\paragraph{Problem 1} %(Collaborated with AAA, BBB, and CCC.)

We'll first prove that $\forall \varepsilon$,
$\sigma = $

\bigskip

\paragraph{Problem 2} (Collaborated with YYY and ZZZ.)

The root is player $6$, and all the other players are his children and the terminated nodes.\\
For $\forall i$, $i \neq 6$, 
suppose player $6$ choose the strategy $\sigma = aH + bT$, $0 \leq a,b \leq 1$.
Then player $i$ choose the strategy $\sigma' = a'H + b'T$,
$0 \leq a', b' \leq 1$.
We have
\begin{align*}
    u_i = (a - b)(a' - b')
\end{align*}
Suppose that $a > b$, 
to maximize $u_i$, we'll make $a' = 1, b' = 0$.
Similarly when $b > a$,
$a' = 0, b' = 1$.\\
And for $a = b = \frac{1}{2}$, $a', b'$ will take arbitrary value.
Therefore, 
\begin{align*}
    T(aH + bT, H) = 1 &\mbox{a > b}\\ 
    T(aH + bT, T) = 1 &\mbox{a < b}\\
    T(aH + bT, a'H + b'T) = 1 &\mbox{a = b, $0 \leq a', b' \leq 1$}\\
    T(X, Y) = 0 &\mbox{otherwise}
\end{align*}
For player 6,
when $a \neq b$, 
it's obvious that $\sigma$ is not the best response to his children's choices. \\  
So we'll only consider $a = b = \frac{1}{2}$.
to find a $z_1, z_2 \ldots z_5$ that $\sigma$ being the best response to them.\\
Clearly, if the opponents of player 6 has higher probability to choose Head or Tail.
He'll choose the other one, so he can get more utility, 
then a pure strategy is his best response. \\
So only when all his opponents have the same probability 
to choose either sides of the coin,
$\sigma$ is the best choice for him. \\
Therefore, $T(\frac{1}{2}H + \frac{1}{2}T) = 1$.
The witness list is all $(z_1, z_2, z_2, z_3, z_4, z_5)$, 
sum of probability that all the children choose Head is equal to 
sum of probability that all the children choose Tail. \\
Then we'll find the Nash Equilibrium top-down.
\begin{align*}
    \sigma_i = a_iH + b_iT, 0 \leq a_i,b_i \leq 1, i \neq 6 \\ 
    \sigma_6 = \frac{1}{2}H + \frac{1}{2}T \\
    \Sigma_{i = 1}^{5} a_i = \Sigma_{i = 1}^{5}b_i
\end{align*}




\bigskip

\paragraph{Problem 3} (No collaborator.)

Your solution goes here.

\bibliographystyle{agsm}

\begin{thebibliography}{99}

\bibitem{OR94}{M. J. Osborne and A. Rubinstein. {\em A course in game theory.} MIT Press, 1994.}

\bibitem{NRTV07}{N. Nisan, T. Roughgarden, E. Tardos, and V. Vazirani (eds). {\em Algorithmic game theory.} Cambridge University Press, 2007. (Available at \url{http://www.cambridge.org/journals/nisan/downloads/Nisan_Non-printable.pdf}.)}

\end{thebibliography}

\end{document}








